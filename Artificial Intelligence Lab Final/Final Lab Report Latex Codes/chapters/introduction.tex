\section{Background}
Cassava is a staple food crop grown in tropical and subtropical regions. It is valued for its high-yield tubers and nutrient-rich leaves. Its resilience in poor soils makes it a vital food source in many low-income regions. However, cassava plants are highly susceptible to several leaf diseases including Cassava Bacterial Blight (CBB), Cassava Brown Streak Disease (CBSD), Cassava Green Mottle (CGM), and Cassava Mosaic Disease (CMD). These diseases disrupt the photosynthetic process and significantly reduce crop yield and quality.

Conventional methods of disease detection involve laboratory testing or expert consultation, both of which are costly and time-consuming. As a result, farmers often lack the means to detect and treat diseases early. With recent advances in machine learning, especially in image classification using deep learning, it is now possible to build systems that can classify crop diseases from images with high accuracy and low cost. This project applies such techniques to develop a cassava leaf disease detection system.

\section{Problem Statement}
In regions like Bangladesh, where cassava is not yet widely cultivated but has strong potential, there is limited infrastructure for disease monitoring. Farmers lack tools for early and accurate identification of leaf diseases. Manual diagnosis is often not viable due to lack of expertise, cost, and time constraints.

This research focuses on the classification of cassava leaf diseases using machine learning models trained on labeled image datasets. The goal is to evaluate various deep learning models to determine which provides the best trade-off between accuracy and computational efficiency, ultimately aiding in timely and scalable disease detection.

\section{Chapter Distribution}
This report is organized into the following chapters:

\begin{itemize}
    \item \textbf{Chapter 1: Introduction} — Introduces the background of the study, identifies the research problem, and outlines the structure of the report.
    
    \item \textbf{Chapter 2: Literature Review} — Summarizes related works from at least eight research papers relevant to crop disease classification, focusing on methods, datasets, and performance metrics.

    \item \textbf{Chapter 3: Methodology} — Describes the overall method used in the project, details the machine learning models implemented, and explains the proposed framework that yielded the best performance.

    \item \textbf{Chapter 4: Experimental Result and Analysis} — Provides a description of the dataset, evaluation metrics, and parameter settings. It also discusses experimental results in detail, including error analysis, limitations, and the social or cultural impact of the proposed solution.

    \item \textbf{Chapter 5: Conclusion and Future Work} — Summarizes the research findings, reflects on the limitations, and suggests directions for future improvements and further study.
\end{itemize}

