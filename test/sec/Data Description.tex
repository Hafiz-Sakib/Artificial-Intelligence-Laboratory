\section{Data Description}
\subsection{Source of Data}
The dataset used for this project comprises approximately 17,000 labeled images of crops and their associated diseases. These images are sourced from publicly available datasets, including agricultural research repositories and academic datasets tailored for crop disease classification. Additionally, the dataset may be supplemented with synthetic data generated through data augmentation techniques to improve model performance and generalization.

\subsection{Type of Data}
The dataset primarily consists of unstructured image data. Each image belongs to one of five distinct classes, representing specific crop diseases. The data is organized in the following format:
\begin{itemize}
    \item \textbf{Input Data}: High-resolution images of crops captured under various environmental conditions.
    \item \textbf{Labels}: Each image is annotated with its corresponding crop disease class, enabling supervised learning.
\end{itemize}

\subsection{Data Preprocessing Requirements}
To ensure optimal model performance and reliability, the dataset will undergo the following preprocessing steps:
\begin{itemize}
    \item \textbf{Image Resizing}: All images will be resized to a uniform dimension to standardize input for the machine learning model.
    \item \textbf{Normalization}: Pixel values will be normalized to a common scale to enhance model convergence during training.
    \item \textbf{Data Augmentation}: Techniques such as rotation, flipping, and cropping will be applied to artificially increase dataset diversity and address class imbalance.
    \item \textbf{Noise Removal}: Images with poor quality, excessive noise, or irrelevant content will be filtered out.
    \item \textbf{Splitting}: The dataset will be divided into training, validation, and test sets to evaluate model performance effectively.
\end{itemize}

By ensuring proper preprocessing and leveraging diverse image data, the project aims to build a robust and accurate crop disease detection system.
