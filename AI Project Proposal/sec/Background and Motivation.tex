\section{Background and Motivation}
Agriculture plays a vital role in sustaining human life and economic development. However, crop diseases remain a persistent challenge, affecting global food production and supply chains. Early and accurate diagnosis of crop diseases is crucial for mitigating losses, but traditional methods rely heavily on human expertise, which can be time-consuming and error-prone. 

The relevance of this problem lies in its potential impact on food security, economic stability, and environmental sustainability. By automating the identification of crop diseases using machine learning, this project seeks to address these challenges, empowering farmers with tools for proactive disease management.

Advancements in artificial intelligence and the availability of large-scale image datasets present a unique opportunity to tackle this issue effectively. The project aims to bridge the gap between traditional agricultural practices and modern technological solutions.

\subsubsection{Literature Review}
Several studies have explored the application of machine learning and computer vision in agriculture. For instance, convolutional neural networks (CNNs) have been widely adopted for image-based disease classification, achieving promising results. However, challenges such as dataset imbalance, variations in environmental conditions, and the need for real-time inference remain areas of concern~\cite{ref1, ref2}.

Opportunities lie in improving model generalization and integrating disease prediction systems with IoT-based agricultural tools for seamless field deployment. Recent research highlights the potential for transfer learning and data augmentation techniques to overcome dataset limitations~\cite{ref3}. These insights form the foundation for this project, which aims to build a robust system addressing these challenges while leveraging existing advancements.
