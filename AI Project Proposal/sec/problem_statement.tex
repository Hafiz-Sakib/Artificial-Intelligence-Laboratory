\section{Problem Statement}
The agricultural sector faces significant challenges due to crop diseases, which lead to reduced yield, economic losses, and food insecurity. Traditional methods of disease diagnosis often rely on visual inspections by experts, which are time-consuming, labor-intensive, and prone to human error. Farmers in remote or resource-limited areas may not have access to expert knowledge, further exacerbating the issue.

This project seeks to address the problem of early and accurate identification of crop diseases using machine learning techniques. By leveraging a dataset of approximately 17,000 crop images across five disease classes, the project aims to develop an automated system capable of detecting and classifying crop diseases. 

The solution will be evaluated based on:
\begin{itemize}
    \item \textbf{Accuracy}: The classification performance of the model in correctly identifying diseases.
    \item \textbf{Speed}: The time taken by the system to process an image and provide a diagnosis.
    \item \textbf{Precision and Recall}: Metrics that ensure the system is not only accurate but also reliable in distinguishing between similar disease classes.
    \item \textbf{Scalability}: The ability of the system to handle new datasets and additional disease categories with minimal retraining.
\end{itemize}

% \begin{figure}[H]
%     \centering
%     \includegraphics[width=0.8\textwidth]{image-20230707-054308.png}
%     \caption{Illustration of crop disease detection challenges and proposed solution~\cite{ref4}.}
%     \label{fig:problem_statement}
% \end{figure}

By addressing these evaluation criteria, the project aims to provide a reliable, scalable, and accessible solution to assist farmers and agricultural stakeholders in managing crop health effectively.
